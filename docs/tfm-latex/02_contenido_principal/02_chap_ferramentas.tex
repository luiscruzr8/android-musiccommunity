\chapter{Entorno e ferramentas de desenvolvemento}
\label{chap:ferramentas}

\lettrine{N}{este} procederáse a explicar as ferramentas que se empregaron para a adopción do antigo proxecto e a posta en marcha do novo.


\section{Adopción do código legacy}
\label{sec:adopcion-legacy}

Para maior contextualización do proxecto, describirase máis en concreto os detalles cos que se implementou o pasado proxecto desenvolto no TFG\cite{TFGLuis}. O backend deste anterior desenvolvemento constaba dun servicio \textit{Java} con \textit{Spring Boot} \cite{SpringBoot} e unha base de datos relacional usando MySQL\cite{mysql}.  Por outro lado tiñamos unha aplicación frontend \textit{ad-hoc} a este servicio, implementada nu framework de desenvolvemento web: Angular\cite{angular}.

Polo tanto, para retomar este proxecto foi necesario empregar diferentes ferramentas de desarrollo, en concreto para seguir o desenvolvemento do servicio backend, implementado en \textit{Java}. Tomouse a decisión de empregar como entorno de desenvolvemento o software de Jet Brains, \textit{IntelliJIDEA}\cite{IntelliJIDEA} xa que permite maior flexibilidad á hora de traballar con diferentes versións de \textit{Java}. Ademáis, para a xestión da base de datos do sistema empregouse \textit{MySQL Workbench} ao ser unha base de datos relacional e estar configurada en concreto para este tipo de base de datos.

O obxectivo é iterar sobre este servicio backend e implementar melloras sobre él, de forma que para o aplicativo web os cambios sexan alleos e transparentes para que poida seguir funcionando sen verse afectado.

\section{Desenvolvemento da nova aplicación}
\label{sec:nova-app}

Unha vez establecidas as ferramentas a empregar para continuar co desenvolvemento do código legacy, púxose en marcha a implementación da aplicación Android. Con esa finalidade empregouse o entorno de desenvolvemento oficial da plataforma: Android Studio \cite{androidStudio}. Pero para modernizar máis o proxecto, decidiuse non empregar \textit{Java} como lenguaxe de programación, senon que \textit{Kotlin}\cite{kotlin}. 