%%%%%%%%%%%%%%%%%%%%%%%%%%%%%%%%%%%%%%%%%%%%%%%%%%%%%%%%%%%%%%%%%%%%%%%%%%%%%%%%
% Preámbulo                                                                    %
%%%%%%%%%%%%%%%%%%%%%%%%%%%%%%%%%%%%%%%%%%%%%%%%%%%%%%%%%%%%%%%%%%%%%%%%%%%%%%%%

\documentclass[11pt,a4paper,titlepage,twoside, openright]{report}

%%%%%%%%%%%%%%%%%%%%%%%%%
%% SELECCIÓN DE IDIOMA %%
%%%%%%%%%%%%%%%%%%%%%%%%%
% Descomenta la línea correspondiente al idioma de la memoria
%\def\idioma{esp}  % Castellano (Español)
\def\idioma{gal} % Galego
%\def\idioma{eng} % English


%%%%%%%%%%%%%%%%%%%%%%%%%%%%%%%%%%%%%
%% DATOS DEL PROYECTO A CONFIGURAR %%
%%%%%%%%%%%%%%%%%%%%%%%%%%%%%%%%%%%%%
\def\titulo{Music Community - Aplicación móbil adicada á xestión de partituras e comunicación de eventos musicais}      % cambiar aquí el título completo
\def\alumno{Luis Fernando Cruz Ramos}                  % nombre del alumno     
\def\directorA{Tiago Manuel Fernandez Caramés}             % nombre del director
%\def\directorB{Nombre del segundo director}    % descomentar aquí está línea y en la portada
                                                % para incluir uno más.
                                                % De haber más añadirlos cambiando la letra por C, D, ...

%%%%%%%%%%%%%%%%%%%%%%%%%%%%%%%%%
%% CARGA DEL FICHERO DE ESTILO %%
%%%%%%%%%%%%%%%%%%%%%%%%%%%%%%%%%
\usepackage{estilo_tfm}


%%%%%%%%%%%%%%%%%%%%%%%%%%%%%%%%%%%%%%%%%
%% PAQUETES QUE PUEDEN SER DE UTILIDAD %%
%%%%%%%%%%%%%%%%%%%%%%%%%%%%%%%%%%%%%%%%%

% \usepackage{alltt}       % para campos alltt, similar a verbatim pero que respecta comandos
% \usepackage{booktabs}     % permite mejorar el aspecto de las tablas con líneas de diferente grosor para el cierre
 \usepackage{enumitem}     % permite personalizar los entornos de lista
% \usepackage{eurofont}    % proporciona el comando \euro
% \usepackage{float}       % permite más control de objetos flotantes (tablas y figuras)
% \usepackage{hhline}      % permite personalizar las liñas horizontales en arrays y tablas
  \usepackage{longtable}   % permite construir tablas que ocupan más de una página
% \usepackage{lscape}      % permite colocar partes del documento en orientación apaisada
% \usepackage{moreverb}    % permite personalizar o entorno verbatim
% \usepackage{multirow}    % permite crear celdas que ocupan varias filas en una tabla
% \usepackage{pdfpages}    % permite insertar ficheros PDF
% \usepackage{rotating}    % permite diferentes tipos de rotaciones para los flotantes
% \usepackage{subcaption}  % permite a inclusión de varias subfiguras en una figura
% \usepackage{tabu}        % permite tablas flexibles
% \usepackage{tabularx}    % permite tablas con columnas de anchura determinada

%%%%%%%%%%%%%%%%%%%%%%%%%%%%%%%%%%%%%%%%%%%%%%%%%%%%%%%%%%%%%%%%%%%%%%%%%%%%%%%%
% CUERPO PRINCIPAL DEL DOCUMENTO                                               %
%%%%%%%%%%%%%%%%%%%%%%%%%%%%%%%%%%%%%%%%%%%%%%%%%%%%%%%%%%%%%%%%%%%%%%%%%%%%%%%%

\begin{document}

 %%%%%%%%%%%%%%%%%%%%%%%%%%%%%%%%%%%%%%%%
 % Preliminares do documento            %
 %%%%%%%%%%%%%%%%%%%%%%%%%%%%%%%%%%%%%%%%

 \begin{titlepage}
\vspace*{-35pt}
\begin{minipage}{.15\linewidth}
    \begin{flushleft}
    \hspace*{-15pt}
      \includegraphics[scale=0.15]{06_imagenes/MUEI.png}
    \end{flushleft}
  \end{minipage}
  \hfill
  \begin{minipage}{.15\linewidth}
    \begin{flushright}
     \includegraphics[scale=0.10]{06_imagenes/euro_inf_master_verde.png}
    \end{flushright} 
  \end{minipage}
 
 \vspace*{30pt}
 
\includegraphics[scale=0.30]{06_imagenes/udc}\\[10pt]
  \hspace*{18pt}\textcolor{udcpink}{{\fontencoding{T1}\fontfamily{phv}\selectfont Facultade de Informática}}\\[5pt]
  \hspace*{18pt}\textcolor{udcgray}{{\fontencoding{T1}\fontfamily{phv}\selectfont Máster Universitario en Enxeñaría en Informática}}\\[5pt]
  
\textcolor{udcpink}{\titlerule[0.5mm]}\\  

  \begin{center}
    {\large TRABALLO FIN DE MÁSTER} \\[25pt]
    \begin{huge}
      \begin{spacing}{1.3}
      \vspace*{10pt}
        \bfseries \titulo
      \end{spacing}
    \end{huge}
  \end{center}
  
\textcolor{udcpink}{\titlerule[0.5mm]}\\[5pt]
  \vfill
  
  \begin{flushright}
    {\large
    \begin{tabular}{ll}
      {\bf Estudiante:} & \alumno \\
      {\bf Dirección:} & \directorA \\
%                      & \directorB \\  % duplicar está línea con los 
                                        % que sean necesarios cambiado la letra
    \end{tabular}}
    
  \end{flushright}
  \rightline{A Coruña, \fecha.}
  
  
\end{titlepage}

\paginablanca
 \thispagestyle{empty}
\mbox{}
\vfill
\hfill
\emph{N} % Rellenar la dedicatoria del documento de haberla
\vfill
\paginablanca
\paginablanca % de no haberla comentar esta línea
 \begin{agradecimientos}

	A miña familia e amigos, que tanto me aguantaron nesta recta final e ao meu titor, que sempre tuvo tanta paciencia conmigo.               % Aquí debe de ir el texto correspondiente a los agradecimientos de haberlos

\end{agradecimientos}
\paginablanca % de no haberlos comentar esta línea
 %%%%%%%%%%%%%%%%%%%%%%%%%%%%%%%%%%%%%%%%%%%%%%%%%%%%%%%%%%%%%%%%%%%%%%%%%%%%%%%%

\begin{abstract}\thispagestyle{empty}
  \blindtext % introducir el texto del resumen aquí,
             % recuerde escribirlo en el mismo idioma que el principal de la memoria

  \vspace*{25pt}
  \begin{abstractAlt}
    \blindtext % introducir el resumen en el idioma alternativo
               % habitualmente debe de ser en inglés
  \end{abstractAlt}
\vspace*{25pt}
\begin{multicols}{2}
\begin{description}
\item [\keywords:] \mbox{} \\[-20pt]
  \blindlist{itemize}[3]    % sutituir este comando por una lista de palabras
                            % deben de ser palabras que describan el TFM lo mejor posible   
\end{description}
\begin{description}
\item [\keywordsAlt:] \mbox{} \\[-20pt]
  \blindlist{itemize}[3]    % al igual que en el caso anterior pero en este caso en el idioma alternativo
                            % (habitualmente en inglés)
\end{description}
\end{multicols}

\end{abstract}
%%%%%%%%%%%%%%%%%%%%%%%%%%%%%%%%%%%%%%%%%%%%%%%%%%%%%%%%%%%%%%%%%%%%%%%%%%%%%%%%
\paginablanca

%%%%%%%%%%%%%%%%%%%%%%%%%%%%%%%%%%%%%%%
 % Índices y numeración en romano     %
 %%%%%%%%%%%%%%%%%%%%%%%%%%%%%%%%%%%%%%
 \pagenumbering{Roman}
 \setcounter{page}{1}
 \bstctlcite{IEEEexample:BSTcontrol}


 \tableofcontents
 \listoffigures
 \listoftables
 \clearpage
 \paginablanca
 
 
 %%%%%%%%%%%%%%%%%%%%%%%%%%%%%%%%%%%%%%%%
 % Capítulos   de la memoria           %
 %%%%%%%%%%%%%%%%%%%%%%%%%%%%%%%%%%%%%%%%
 
 % Numeración arábiga
 \pagenumbering{arabic}
 \setcounter{page}{1}

 \chapter{Introducción}
\label{chap:introduccion}

\lettrine{N}{este} primeiro capítulo da memoria abordaráse a introducción á problemática a tratar. Ademáis, explicarase a motivación para levar a cabo o desenvolvemento do proxecto e por últimos comentaránse os obxetivos que se esperan acadar.

***********************************
A continuación se muestran un ejemplo de como referencias una sección 
en concreto del documento como puede ser la de objetivos del trabajo. 
En el presente documento, esta sección es la Sección \ref{sec:objetivos}
que podemos ver en la página \pageref{sec:objetivos}.

Por último dentro de este primer capítulo de ejemplo, se incluyen un 
par de citas~\cite{RAE,RAGA, van2005student} bibliográficas a modo 
de ejemplo. En este sentido, si los o las estudiantes piensan en 
escribir está memoria en en overleaf \cite{Overleaf}, se recomienda el 
consultar la integración con Mendeley \cite{Mendeley} ya que les puede 
facilitar el mantenimiento de las referencias.
***************************************

\section{Motivación}
\label{sec:motivación}

Hoxe en día, existen diferentes comunidades en gremios, así como acontece co conxuntos dos músicos: un grupo nicho dentro da sociedade que buscan desenvolverse neste ámbito pero moitas veces resulta difícil debido á ampla cantidade de información que circula polas redes sociais e distintos foros. Aínda que cada vez esta información sexa personalizada, sigue habendo datos que pode pasar desapercibidos para certos colectivos, como pode ser o caso dos músicos, que poder perder información debido ao gran volume de datos na actualidade.

Tras consultar e compartir pensamentos con varios músicos e expertos no tema, analizouse que clase de información eles considerarían de relevancia e virían necesaria nunha proposta de aplicación enfocada e destinada ó colectivo de músicos, ou persoas interesadas en seguir eventos  musicais de todo tipo. Houbo concenso xeral na necesidade de dar a coñecer eventos locais ou pequenos, ou anuncios buscando xente para tocar en concertos que moitas veces pasaban desapercibidos en outras redes sociais máis grandes. Entre outras ideas, tamen mencionaban a necesidade de poder discutir en foros de discusións nos que expoñer e comentar o seu traballo, con outra xente de experiencia similar a eles, que calquera crítica constructiva lle sería de utilidade xa que existen moitos foros nos que persoas alleas ou pouco expertos no tema tenden a participar en debates sen fundamentos ou xustificacións non válidas.
En moitos casos, os encuestados compartían unha opinón xeral de ter un lugar no que almacenar todas as súas partituras e que foran accesibles desde calquer lugar. 

Polo tanto xurde así a idea desta aplicación: \textbf{\textit{Music Community}}, que foi desenvolvido durante o meu proxecto fin de grao. Polo que, para o deselvovemento deste proxecto fin de máster, levarase a cabo a transformación de dito proxecto, no seu momento unha plataforma web, a un entorno móbil.  Partirase así dun sistema xa establecido, que consta dun servicio REST e unha aplicaicón web, que non se verán afectados polo desenvolvemento deste proxecto salvo unha modificación que se fará para os dispositivos móbiles: as notificacións push. Decidiuse actualizalo proxecto xa que as tendencias cambian no tempo, e debemos adaptarnos as novas tecnoloxías. Isto representa un novo desafío a nivel académico e persoal, no que asentar os coñecementos obtidos no mestrado, realizando a transformación dunha aplicación legacy nun proxecto actualizado.

\section{Estado do arte}
\label{sec:estado-arte}

No meu proxecto de grao, fixérase unha comparativa con diferentes aplicacións no ámbito web. Para este proxecto avaliaránse, por consecuente, aplicacións no ámbito móbil que compartan funcionalidades similares ás des critas previamente para este proxecto. Se ben, moitas das apps citadas neste apartado, cumpren a funcionalidade recollida no proxecto, moitas delas son de pago ou tipo \textit{freemium} , para as que fai falta realizar algún tipo de pago para usalas ao completo. Entre estas podemos destacar:

\begin{itemize}
	\item \textbf{\textit{MuseScore: partitura}} 
	
	\begin{figure}[h!]
		\centering
		\includegraphics[scale=0.3]{imaxes/intro/musescore-app}
		\caption{Logo 'MuseScore: partitura' (\url{https://play.google.com/store/apps/details?id=com.musescore.playerlite&hl=es&pli=1})}
	\end{figure}
	
	Esta é unha app Android que permite a xestión, visualización e reproducción de partituras. Este exemplo non profundiza no apartado social da app, ademáis de ser unha app  \textit{freemium}, moitos usuarios gratuitos queixanse bastante do funcionamento xeral da versión gratuita.
	A principal ventaxa desta app é o seu catálogo e a variedade de partituras, e a principal desventaxa é o baixo rendemento da súa versión gratuita, empeorando a experiencia do usuarios básicos. 

\end{itemize}

Por outro lado, temos aplicacións que cumplen no ámbito social, pero que chegan a ser de temática tan xeral que non cubren as necesidades que neste proxecto se van levar a cabo.

\begin{itemize}
	\item \textbf{\textit{Facebook}} 
	
	A opción máis coñecida, que é capaz de cumplir con todos os puntos explicados anteriormente. Ademáis permite iniciar  debates nas publicacións, compartir eventos de interés, publicar anuncios\dots 
	Independientemente, esta app ten moitísimas funcionalidades extra que ensombrecen a experencia de usuario para determinados colectivos.
	
	\item \textbf{\textit{X (antigamente coñecida como Twitter)}} 
	
	Outra opción bastante coñecida  polos usuarios á hora de manter fíos de discusión e compartir novedades de forma rápida, moi similar ao boca a boca. En calquer caso, esta app tamén empeorou a súa experiencia de usuario coa reciente integración de publicidade e cambio de dirección.
	
\end{itemize}

Con todo, dentro das alternativas mencionadas non se atoparon exemplos que cubran as funcionalidades descritas previamente: unha app móbil que lle permita aos músicos unha forma de darse a conocer e recibir \textit{retroalimentación} das súas obras e incluso que lles permita estar enterados de eventos, anuncios ou discusións de actualidade.

\section{Objetivos}
\label{sec:objetivos}
Una de las secciones más habituales dentro de la memoria es la correspondiente a los objetivos del proyecto, los cuales se suelen separar en objetivos generales y objetivos concretos.  

\subsection{Objetivos principales}

\begin{enumerate}[label={\textbf{Objetivo \arabic*:}},leftmargin=2.5cm,labelindent=\parindent]
    \item Primer objetivo del proyecto
    \item Segundo objetivo del proyecto
    \item Los siguientes si los hubiere \ldots
\end{enumerate}

\subsection{Objetivos concretos}
\begin{enumerate}[label={\textbf{Objetivo Concreto \arabic*:}},leftmargin=4.5cm,labelindent=\parindent]
    \item Primero
    \item Segundo 
    \item Y así sucesivamente \ldots
\end{enumerate}


\section{Estructura de la memoria}
Otro de los elementos típicos dentro del primer capítulo de las memorias es designar las estructura de la misma con una brevísima  explicación de que se puede encontrar en cada capítulo.

\begin{enumerate}[label={Capítulo \arabic*},leftmargin=2.5cm]
\item Este mismo capítulo.
\item El segundo de los capítulo centra \ldots
\end{enumerate}


 \chapter{Entorno e ferramentas de desenvolvemento}
\label{chap:ferramentas}

\lettrine{N}{este} procederáse a explicar as ferramentas que se empregaron para a adopción do antigo proxecto e a posta en marcha do novo.


\section{Adopción do código legacy}
\label{sec:adopcion-legacy}

Para maior contextualización do proxecto, describirase máis en concreto os detalles cos que se implementou o pasado proxecto desenvolto no TFG\cite{TFGLuis}. O backend deste anterior desenvolvemento constaba dun servicio \textit{Java} con \textit{Spring Boot} \cite{SpringBoot} e unha base de datos relacional usando MySQL\cite{mysql}.  Por outro lado tiñamos unha aplicación frontend \textit{ad-hoc} a este servicio, implementada nu framework de desenvolvemento web: Angular\cite{angular}.

Polo tanto, para retomar este proxecto foi necesario empregar diferentes ferramentas de desarrollo, en concreto para seguir o desenvolvemento do servicio backend, implementado en \textit{Java}. Tomouse a decisión de empregar como entorno de desenvolvemento o software de Jet Brains, \textit{IntelliJIDEA}\cite{IntelliJIDEA} xa que permite maior flexibilidad á hora de traballar con diferentes versións de \textit{Java}. Ademáis, para a xestión da base de datos do sistema empregouse \textit{MySQL Workbench} ao ser unha base de datos relacional e estar configurada en concreto para este tipo de base de datos.

O obxectivo é iterar sobre este servicio backend e implementar melloras sobre él, de forma que para o aplicativo web os cambios sexan alleos e transparentes para que poida seguir funcionando sen verse afectado.

\section{Desenvolvemento da nova aplicación}
\label{sec:nova-app}

Unha vez establecidas as ferramentas a empregar para continuar co desenvolvemento do código legacy, púxose en marcha a implementación da aplicación Android. Con esa finalidade empregouse o entorno de desenvolvemento oficial da plataforma: Android Studio \cite{androidStudio}. Pero para modernizar máis o proxecto, decidiuse non empregar \textit{Java} como lenguaxe de programación, senon que \textit{Kotlin}\cite{kotlin}. 
 \chapter{Ejemplos de uso}
\label{chap:demo}

\lettrine{E}{ntre} la introducción y las conclusiones, este documento 
debiera de contener todos los detalles necesario para explicar el 
desarrollo del proyecto en su totalidad. 

En las siguientes secciones se mostrará el uso de algunos de los 
elementos más habituales dentro de los TFMs o que con mayor recurrencia 
se utilizan\footnote{Por ejemplo, la inclusión de notas al pie de página}.


\section{Inclusión de imágenes}\label{sec:incluirImagen}
Uno de los elementos que con total seguridad será necesario es la 
inclusión de imágenes que apoyen la explicación del texto con un apoyo visual.
Estos elementos junto con las tablas, qué se abordan en el apartado
\ref{sec:tablas} en la página \pageref{sec:tablas}, es lo que se conoce como flotantes ya que
no tienen un lugar predeterminado si no que \LaTeX intenta colocarlos de la mejor 
manera posible dentro de nuestro texto.

Por ejemplo, piense que quiere incluir una figura como la que aparece en la 
figura~\ref{fig:ejemplo} dentro de la página~\pageref{fig:ejemplo}. Si se hace 
de está manera, \LaTeX intentará ubicar  la si bien la puede recolocar a medida 
que el documento crezca.

\begin{figure}[hp!] % opciones de forzar lo má arriba posible (h) y que forme un párrafo(p)
  \centering
  \includegraphics[width=0.75\textwidth]{06_imagenes/udc.png}
  \caption{Todas las imágenes deberían de llevar un pie descriptivo}
  \label{fig:ejemplo}
\end{figure}

Como recomendación general, es conveniente el guardar todos los ficheros de 
imágenes en el directorio (carpeta) 
\texttt{06\_imagenes}.

\subsection{Inclusión de varias sub-imaxes}\label{subsec:imagenes}

En el caso más que provable de que sea necerasia la inclusión de más de una imagen, 
existen diferentes alternativas. Dentro de las posibilidades una es el uso de 
subfiguras como se puede ver en el ejmplo de la figura~\ref{fig:ejemplo-subfiguras}.
La principal ventaja es que, de este modo, se pueden refecrencias cada una de las subfiguras 
como en este caso la imagen de la izquierda (figura~\ref{fig:subfigura-rotada}) y la 
de las derecha (figura~\ref{fig:subfigura-deformada}). 

\begin{figure}[hp!]
  \centering
  \begin{subfigure}[c]{0.3\textwidth}
    \includegraphics[angle=45,width=\textwidth]{06_imagenes/MUEI.png}
    \caption{Logo del MUEI}
    \label{fig:subfigura-rotada}
  \end{subfigure}
  \hspace{0.1\textwidth}
  \begin{subfigure}[c]{0.3\textwidth}
    \includegraphics[width=\textwidth,height=3cm]{06_imagenes/euro_inf_master_verde.png}
    \caption{Logo Euro-INF}
    \label{fig:subfigura-deformada}
  \end{subfigure}
  \caption{Logos relacionados con el MUEI que han sido usados como ejemplos de uso para las figuras}
  \label{fig:ejemplo-subfiguras}
\end{figure}

\section{Inclusión de tablas}\label{sec:tablas}

Otro de los elementos que se utilizarán con total seguridad dentro 
del documento son las tablas. Estas al igual que las imágenes (apartado 
\ref{sec:incluirImagen} serán ubicadas en el sitio que \LaTeX considere más adecuado 
para la maquetación. Si la ubicamos como en el ejemplo de la Tabla~\ref{tab:ejemplo} en 
la página~\pageref{tab:ejemplo}), estas se podrán referenciar sin problemas. 

\begin{table}[hp!]
  \centering
  \rowcolors{2}{white}{udcgray!15}
  \begin{tabular}{l|c} % Posibles valores izquierda (l), centro (c), derecha (r)
  \rowcolor{ficblue!50}
  \textcolor{white}{\textbf{Cabecera}} & \textcolor{white}{\textbf{Cabecera Valores}} \\\hline
  \textit{Fila de ejemplo} & Valor asociado \\
  \textit{Fila de ejemplo} & Valor asociado \\
  \textit{Fila de ejemplo} & Valor asociado \\
  \textit{Fila de ejemplo} & Valor asociado \\
  \multicolumn{2}{c}{Ejemplo de combinación de dos columnas en una}\\
  \textit{Fila de ejemplo} & Valor asociado \\
  \end{tabular}
  \caption{Leyenda de la tabla que se ubica en la parte inferior}
  \label{tab:ejemplo}
\end{table}


Así mismo, en el caso de tener muchos resultados es posible que la tabla sea tan grande 
que no entre en una sola página. Para las tablas excesivamente grandes se pude usar 
el paquete \texttt{longtable}. Este nos ayudará a gestionar este problema como se ve en 
en el ejemplo de la Tabla \ref{tab:longtable}.

\rowcolors{2}{white}{udcgray!15}
\begin{longtable}{l|r|c}
  \caption{Título de una tabla larga}
  \label{tab:longtable} \\

  \rowcolor{ficblue!50}
  \textcolor{white}{\textbf{Texto alineado a la}} & \textcolor{white}{\textbf{Texto alineado a la}} & \textcolor{white}{\textbf{Cantidad}} \\\hline
  \endfirsthead

  \multicolumn{3}{c}{\tablename\ \thetable{} -- {\textit{Continuación}}} \\
  \rowcolor{ficblue!50}
  \textcolor{white}{\textbf{Texto alineado a la }} & \textcolor{white}{\textbf{Texto alineado a la}} & \textcolor{white}{\textbf{Cantidad}} \\\hline
  \endhead

  \multicolumn{3}{r}{\textit{Continúa en la página siguiente.}} \\
  \endfoot

  \endlastfoot

  Izquierda & Derecha & 1.234,5678 \\
  Izquierda & Derecha & 1.234,5678 \\
  Izquierda & Derecha & 1.234,5678 \\
  Izquierda & Derecha & 1.234,5678 \\
  Izquierda & Derecha & 1.234,5678 \\
  Izquierda & Derecha & 1.234,5678 \\
  Izquierda & Derecha & 1.234,5678 \\
  Izquierda & Derecha & 1.234,5678 \\
  Izquierda & Derecha & 1.234,5678 \\
  Izquierda & Derecha & 1.234,5678 \\
  Izquierda & Derecha & 1.234,5678 \\
  Izquierda & Derecha & 1.234,5678 \\
  Izquierda & Derecha & 1.234,5678 \\
  Izquierda & Derecha & 1.234,5678 \\
  Izquierda & Derecha & 1.234,5678 \\
  Izquierda & Derecha & 1.234,5678 \\
  Izquierda & Derecha & 1.234,5678 \\
  Izquierda & Derecha & 1.234,5678 \\
  Izquierda & Derecha & 1.234,5678 \\
  Izquierda & Derecha & 1.234,5678 \\
  Izquierda & Derecha & 1.234,5678 \\
  Izquierda & Derecha & 1.234,5678 \\
  Izquierda & Derecha & 1.234,5678 \\
  Izquierda & Derecha & 1.234,5678 \\
  Izquierda & Derecha & 1.234,5678 \\
  Izquierda & Derecha & 1.234,5678 \\
  Izquierda & Derecha & 1.234,5678 \\
  Izquierda & Derecha & 1.234,5678 \\
  Izquierda & Derecha & 1.234,5678 \\
  Izquierda & Derecha & 1.234,5678 \\
  Izquierda & Derecha & 1.234,5678 \\
  Izquierda & Derecha & 1.234,5678 \\
  Izquierda & Derecha & 1.234,5678 \\
  Izquierda & Derecha & 1.234,5678 \\
  Izquierda & Derecha & 1.234,5678 \\
  Izquierda & Derecha & 1.234,5678 \\
  Izquierda & Derecha & 1.234,5678 \\
  Izquierda & Derecha & 1.234,5678 \\
  Izquierda & Derecha & 1.234,5678 \\
  Izquierda & Derecha & 1.234,5678 \\
  Izquierda & Derecha & 1.234,5678 \\
  Izquierda & Derecha & 1.234,5678 \\
  Izquierda & Derecha & 1.234,5678 \\
  Izquierda & Derecha & 1.234,5678 \\
  Izquierda & Derecha & 1.234,5678 \\
  Izquierda & Derecha & 1.234,5678 \\
  Izquierda & Derecha & 1.234,5678 \\
  Izquierda & Derecha & 1.234,5678 \\

\end{longtable}


UN elemento a resaltar es que las tablas largas no se consideran del todo flotantes y por 
lo tanto no entran en la maquetación. Tenga cuidado porque una tabla más pequeña puede 
aparecer más tarde que una tabla larga que se encuentre en un punto posterior del texto.

\section{Código fuente}

Habrá veces que sea necesario también la inclusión de código fuente en el texto para 
ello se puede usar el entorno \texttt{lstlisting}. Vease el siguiente ejemplo

\begin{lstlisting}[language=C]
#include <stdio.h>
#define N 10

int main()
{
  int i;
  
  /*Ejemplo de código que imprime 10 veces 
    el mensaje más famoso de la informática
  */

  for (i = 0; i < N; i++)
  {
    printf("Hola mundo!");
  }

  return 0;
}
\end{lstlisting}

\begin{lstlisting}[language=Python]
def main():

  #Mismo programa que el anterior pero en Python
  for _ in range(10)
      print(f"Hola Mundo!")
}

if __name__ = __main__:
    main()   
\end{lstlisting}

\section{Acrónimos y Glosario}

En \texttt{04\_terminos/01\_acronimos.tex}, se pueden editar los acrónimos 
o abreviaturas que se van a utilizar. Para hacer uso de los que 
se hayan defino simplemente se necesitará usar las instrucciones 
\texttt{acrlong} para la versión larga del término y \texttt{acrshort}
para obtener el acrónimo. Es de destacar que, la primera vez que se 
referencia a algo en el texto, es necesario poner el nombre completo y 
la definición del corto. Para no tener que invocar ambas instruicciones 
se puede usar la instrucción \texttt{acrfull}, que nos dará el resultado
deseado. Por ejemplo, la primera vez que llamamos al máster deberíamos referirnos
a el como \acrfull{muei}. A partir de ese punto podría mos usar 
\acrlong{muei} si quisiéramos el nombre completo, o bien \acrshort{muei}
para referirnos por el acrónimo. Un punto a destacare es el hecho de que sólo 
los acrónimos que se usan en el documento aparecen en la relación final.
Por ejemplo, el acrónimo de API está definido en el fichero pero no lo 
se utilizado y por lo tanto no está en el PDF resultante

En cuanto a los términos del glosario, serán las definiciones de los
términos técnicos como puede ser un bytecode o un hash. Además también 
se debieran de incluir aquellos términos relacionados con el problema 
de especial relevancia.
Por ejemplo, piense en que esta realizando una  aplicación para la gestión 
del máster, un término que a cualquier lector tendrái que explicarsele es
el significado de \gls{credito} en el marco de la enseñanza universitaria.
Estos términos estarán definidos en \texttt{05\_terminos/02\_glo\-sa\-rio.tex} 
y se pueden invocar con la orden \texttt{gls} (por ejemplo, \gls{credito}) 
o bien \texttt{Gls} ( \Gls{credito}). Igual que los acrónimos, aquellos 
términos que no se usen en el texto no se incluirán en el listado final.

\section{Errores comunes}

Existen algunos errores que son sencillos de detectar pero que, en muchas ocasiones, 
al autor se le pasan inadvertidos por culpa de una lectura un tanto descuidada. 

\begin{itemize}
    \item Errores con las comillas. El autor queriendo escribir algo entrecomillado 
    como ``ejemplo de entrecomillado''. Acaba escribiendo algo similar a ''error 
    de entrecomillado'' o "error de entrecomillado". En ambos casos se debe que las 
    comillas se han aplicado de manera incorrecta siendo la forma correcta 
    \texttt{``ejemplo de entrecomillado''}.
    \item Errores debido la posición de barras bajas o comillas. Un autor despistado 
    puede acabar escribiendo $06_imagenes$ o algo peor en el párrafo en lugar de 
    06\_imagenes. El problema  viene de que el caracter \_ y el carácter " debieran 
    de escaparse con la barra ya que de otra manera tiene el significado de subindice 
    o superindice, respectivamente.
\end{itemize}
%%%%%%%%%%%%%%%%%%%%%%%%%%%%%%%%%%%%%%%%%%
%%%%%%%%%%%%%%%%%%%%%%%%%%%%%%%%%%%%%%%%%%
% INTRODUCIR EL RESTO DEL CONTENIDO AQUÍ %
%%%%%%%%%%%%%%%%%%%%%%%%%%%%%%%%%%%%%%%%%%
%%%%%%%%%%%%%%%%%%%%%%%%%%%%%%%%%%%%%%%%%%
%\include{02_contenido_principal/...}
 \chapter{Conclusiones}
\label{chap:conclusions}

\lettrine{Ú}{ltimo} o penúltimo capítulo da memoria, dependiendo de si los Futuros Desarrollos se incluyen como un capítulo adicional.
En este se deberá de presentar la situación final del trabajo en la que se desgranen las lecciones aprendidas así como la relación 
con las competencias del título y el carácter profesional del mismo.

\Blindtext[10]


 %%%%%%%%%%%%%%%%%%%%%%%%%%%%%%%%%%%%%%%%
 % Anejos                               %
 %%%%%%%%%%%%%%%%%%%%%%%%%%%%%%%%%%%%%%%%

 \appendix
 \appendixpage
 \chapter{Material adicional}
\label{chap:adicional}

\lettrine{E}{jemplo} de capítulo con formato de apéndice. El objetivo de este tipo de capítulo 
es incluir información adicional complementaria o bien elementos de consulta que por su tamaño no 
son adecuados dentro del texto principal, por ejemplo, un diagrama de clases particularmente grande.

\section{Ejemplo de apartado en anejos}\label{anexo:ejemplo}
\Blindtext

 %%%%%%%%%%%%%%%%%%%%%%%%%%%%%%%%%%%%%%%%%
%   INTRODUCIR EL RESTO DE ANEJOS AQUÍ   %
%%%%%%%%%%%%%%%%%%%%%%%%%%%%%%%%%%%%%%%%%%
%\include{04_anexos/...}

%%%%%%%%%%%%%%%%%%%%%%%%%%%%%%%%%%%%%%%%
% Acrónimos y Glosarios y bibliografía %
%%%%%%%%%%%%%%%%%%%%%%%%%%%%%%%%%%%%%%%%
\printglossary[type=\acronymtype,title=\acronymHeader]
\printglossary[title=\glossaryHeader]
%%%%%%%%%%%%%%%%%%%%%%%%%%%%%%%%%%%%%%%%%%%%%%%%%%%%%%%%%%%%%%%%%%%%%%%%%%%%%%%%
% Objetivo: Lista de siglas, abreviaturas, acrónimos, etc.                     %
%%%%%%%%%%%%%%%%%%%%%%%%%%%%%%%%%%%%%%%%%%%%%%%%%%%%%%%%%%%%%%%%%%%%%%%%%%%%%%%%
\newacronym{api}{\textit{API}}{\textit{Application Programming Interfaces}}
\newacronym{muei}{MUEI}{Máster Universitario de Enxeñaría en Informática}

%%%%%%%%%%%%%%%%%%%%%%%%%%%%%%%%%%%%%%%%%%%%%%%%%%%%%%%%%%%%%%%%%%%%%%%%%%%%%%%%
% Objetivo: Lista de términos ténicos empleados en el documento                %
%           junto con sus definiciones.                                        %
%%%%%%%%%%%%%%%%%%%%%%%%%%%%%%%%%%%%%%%%%%%%%%%%%%%%%%%%%%%%%%%%%%%%%%%%%%%%%%%%

\newglossaryentry{credito}{
  name=Crédito,
  description={Un crédito en el ámbito de la educación universitaria hace referencia a un conjunto de horas en una materia. Dicho conjunto de horas incluye tanto el trabajo realizado en el aula, como el trabajo realizado en casa por parte de los alumnos}
}

%%%%%%%%%%%%%%%%%%%%%%%%%%%%%%%%%%%%%%%
% Bibliografía                        %
%%%%%%%%%%%%%%%%%%%%%%%%%%%%%%%%%%%%%%%
 \bibliographystyle{IEEEtranN}
 \bibliography{\bibconfig,05_bibliografia/bibliografia}
 \clearpage
 
\end{document}

%%%%%%%%%%%%%%%%%%%%%%%%%%%%%%%%%%%%%%%%%%%%%%%%%%%%%%%%%%%%%%%%%%%%%%%%%%%%%%%%
