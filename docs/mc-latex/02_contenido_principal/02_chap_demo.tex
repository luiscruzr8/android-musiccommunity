\chapter{Ejemplos de uso}
\label{chap:demo}

\lettrine{E}{ntre} la introducción y las conclusiones, este documento 
debiera de contener todos los detalles necesario para explicar el 
desarrollo del proyecto en su totalidad. 

En las siguientes secciones se mostrará el uso de algunos de los 
elementos más habituales dentro de los TFMs o que con mayor recurrencia 
se utilizan\footnote{Por ejemplo, la inclusión de notas al pie de página}.


\section{Inclusión de imágenes}\label{sec:incluirImagen}
Uno de los elementos que con total seguridad será necesario es la 
inclusión de imágenes que apoyen la explicación del texto con un apoyo visual.
Estos elementos junto con las tablas, qué se abordan en el apartado
\ref{sec:tablas} en la página \pageref{sec:tablas}, es lo que se conoce como flotantes ya que
no tienen un lugar predeterminado si no que \LaTeX intenta colocarlos de la mejor 
manera posible dentro de nuestro texto.

Por ejemplo, piense que quiere incluir una figura como la que aparece en la 
figura~\ref{fig:ejemplo} dentro de la página~\pageref{fig:ejemplo}. Si se hace 
de está manera, \LaTeX intentará ubicar  la si bien la puede recolocar a medida 
que el documento crezca.

\begin{figure}[hp!] % opciones de forzar lo má arriba posible (h) y que forme un párrafo(p)
  \centering
  \includegraphics[width=0.75\textwidth]{06_imagenes/udc.png}
  \caption{Todas las imágenes deberían de llevar un pie descriptivo}
  \label{fig:ejemplo}
\end{figure}

Como recomendación general, es conveniente el guardar todos los ficheros de 
imágenes en el directorio (carpeta) 
\texttt{06\_imagenes}.

\subsection{Inclusión de varias sub-imaxes}\label{subsec:imagenes}

En el caso más que provable de que sea necerasia la inclusión de más de una imagen, 
existen diferentes alternativas. Dentro de las posibilidades una es el uso de 
subfiguras como se puede ver en el ejmplo de la figura~\ref{fig:ejemplo-subfiguras}.
La principal ventaja es que, de este modo, se pueden refecrencias cada una de las subfiguras 
como en este caso la imagen de la izquierda (figura~\ref{fig:subfigura-rotada}) y la 
de las derecha (figura~\ref{fig:subfigura-deformada}). 

\begin{figure}[hp!]
  \centering
  \begin{subfigure}[c]{0.3\textwidth}
    \includegraphics[angle=45,width=\textwidth]{06_imagenes/MUEI.png}
    \caption{Logo del MUEI}
    \label{fig:subfigura-rotada}
  \end{subfigure}
  \hspace{0.1\textwidth}
  \begin{subfigure}[c]{0.3\textwidth}
    \includegraphics[width=\textwidth,height=3cm]{06_imagenes/euro_inf_master_verde.png}
    \caption{Logo Euro-INF}
    \label{fig:subfigura-deformada}
  \end{subfigure}
  \caption{Logos relacionados con el MUEI que han sido usados como ejemplos de uso para las figuras}
  \label{fig:ejemplo-subfiguras}
\end{figure}

\section{Inclusión de tablas}\label{sec:tablas}

Otro de los elementos que se utilizarán con total seguridad dentro 
del documento son las tablas. Estas al igual que las imágenes (apartado 
\ref{sec:incluirImagen} serán ubicadas en el sitio que \LaTeX considere más adecuado 
para la maquetación. Si la ubicamos como en el ejemplo de la Tabla~\ref{tab:ejemplo} en 
la página~\pageref{tab:ejemplo}), estas se podrán referenciar sin problemas. 

\begin{table}[hp!]
  \centering
  \rowcolors{2}{white}{udcgray!15}
  \begin{tabular}{l|c} % Posibles valores izquierda (l), centro (c), derecha (r)
  \rowcolor{ficblue!50}
  \textcolor{white}{\textbf{Cabecera}} & \textcolor{white}{\textbf{Cabecera Valores}} \\\hline
  \textit{Fila de ejemplo} & Valor asociado \\
  \textit{Fila de ejemplo} & Valor asociado \\
  \textit{Fila de ejemplo} & Valor asociado \\
  \textit{Fila de ejemplo} & Valor asociado \\
  \multicolumn{2}{c}{Ejemplo de combinación de dos columnas en una}\\
  \textit{Fila de ejemplo} & Valor asociado \\
  \end{tabular}
  \caption{Leyenda de la tabla que se ubica en la parte inferior}
  \label{tab:ejemplo}
\end{table}


Así mismo, en el caso de tener muchos resultados es posible que la tabla sea tan grande 
que no entre en una sola página. Para las tablas excesivamente grandes se pude usar 
el paquete \texttt{longtable}. Este nos ayudará a gestionar este problema como se ve en 
en el ejemplo de la Tabla \ref{tab:longtable}.

\rowcolors{2}{white}{udcgray!15}
\begin{longtable}{l|r|c}
  \caption{Título de una tabla larga}
  \label{tab:longtable} \\

  \rowcolor{ficblue!50}
  \textcolor{white}{\textbf{Texto alineado a la}} & \textcolor{white}{\textbf{Texto alineado a la}} & \textcolor{white}{\textbf{Cantidad}} \\\hline
  \endfirsthead

  \multicolumn{3}{c}{\tablename\ \thetable{} -- {\textit{Continuación}}} \\
  \rowcolor{ficblue!50}
  \textcolor{white}{\textbf{Texto alineado a la }} & \textcolor{white}{\textbf{Texto alineado a la}} & \textcolor{white}{\textbf{Cantidad}} \\\hline
  \endhead

  \multicolumn{3}{r}{\textit{Continúa en la página siguiente.}} \\
  \endfoot

  \endlastfoot

  Izquierda & Derecha & 1.234,5678 \\
  Izquierda & Derecha & 1.234,5678 \\
  Izquierda & Derecha & 1.234,5678 \\
  Izquierda & Derecha & 1.234,5678 \\
  Izquierda & Derecha & 1.234,5678 \\
  Izquierda & Derecha & 1.234,5678 \\
  Izquierda & Derecha & 1.234,5678 \\
  Izquierda & Derecha & 1.234,5678 \\
  Izquierda & Derecha & 1.234,5678 \\
  Izquierda & Derecha & 1.234,5678 \\
  Izquierda & Derecha & 1.234,5678 \\
  Izquierda & Derecha & 1.234,5678 \\
  Izquierda & Derecha & 1.234,5678 \\
  Izquierda & Derecha & 1.234,5678 \\
  Izquierda & Derecha & 1.234,5678 \\
  Izquierda & Derecha & 1.234,5678 \\
  Izquierda & Derecha & 1.234,5678 \\
  Izquierda & Derecha & 1.234,5678 \\
  Izquierda & Derecha & 1.234,5678 \\
  Izquierda & Derecha & 1.234,5678 \\
  Izquierda & Derecha & 1.234,5678 \\
  Izquierda & Derecha & 1.234,5678 \\
  Izquierda & Derecha & 1.234,5678 \\
  Izquierda & Derecha & 1.234,5678 \\
  Izquierda & Derecha & 1.234,5678 \\
  Izquierda & Derecha & 1.234,5678 \\
  Izquierda & Derecha & 1.234,5678 \\
  Izquierda & Derecha & 1.234,5678 \\
  Izquierda & Derecha & 1.234,5678 \\
  Izquierda & Derecha & 1.234,5678 \\
  Izquierda & Derecha & 1.234,5678 \\
  Izquierda & Derecha & 1.234,5678 \\
  Izquierda & Derecha & 1.234,5678 \\
  Izquierda & Derecha & 1.234,5678 \\
  Izquierda & Derecha & 1.234,5678 \\
  Izquierda & Derecha & 1.234,5678 \\
  Izquierda & Derecha & 1.234,5678 \\
  Izquierda & Derecha & 1.234,5678 \\
  Izquierda & Derecha & 1.234,5678 \\
  Izquierda & Derecha & 1.234,5678 \\
  Izquierda & Derecha & 1.234,5678 \\
  Izquierda & Derecha & 1.234,5678 \\
  Izquierda & Derecha & 1.234,5678 \\
  Izquierda & Derecha & 1.234,5678 \\
  Izquierda & Derecha & 1.234,5678 \\
  Izquierda & Derecha & 1.234,5678 \\
  Izquierda & Derecha & 1.234,5678 \\
  Izquierda & Derecha & 1.234,5678 \\

\end{longtable}


UN elemento a resaltar es que las tablas largas no se consideran del todo flotantes y por 
lo tanto no entran en la maquetación. Tenga cuidado porque una tabla más pequeña puede 
aparecer más tarde que una tabla larga que se encuentre en un punto posterior del texto.

\section{Código fuente}

Habrá veces que sea necesario también la inclusión de código fuente en el texto para 
ello se puede usar el entorno \texttt{lstlisting}. Vease el siguiente ejemplo

\begin{lstlisting}[language=C]
#include <stdio.h>
#define N 10

int main()
{
  int i;
  
  /*Ejemplo de código que imprime 10 veces 
    el mensaje más famoso de la informática
  */

  for (i = 0; i < N; i++)
  {
    printf("Hola mundo!");
  }

  return 0;
}
\end{lstlisting}

\begin{lstlisting}[language=Python]
def main():

  #Mismo programa que el anterior pero en Python
  for _ in range(10)
      print(f"Hola Mundo!")
}

if __name__ = __main__:
    main()   
\end{lstlisting}

\section{Acrónimos y Glosario}

En \texttt{04\_terminos/01\_acronimos.tex}, se pueden editar los acrónimos 
o abreviaturas que se van a utilizar. Para hacer uso de los que 
se hayan defino simplemente se necesitará usar las instrucciones 
\texttt{acrlong} para la versión larga del término y \texttt{acrshort}
para obtener el acrónimo. Es de destacar que, la primera vez que se 
referencia a algo en el texto, es necesario poner el nombre completo y 
la definición del corto. Para no tener que invocar ambas instruicciones 
se puede usar la instrucción \texttt{acrfull}, que nos dará el resultado
deseado. Por ejemplo, la primera vez que llamamos al máster deberíamos referirnos
a el como \acrfull{muei}. A partir de ese punto podría mos usar 
\acrlong{muei} si quisiéramos el nombre completo, o bien \acrshort{muei}
para referirnos por el acrónimo. Un punto a destacare es el hecho de que sólo 
los acrónimos que se usan en el documento aparecen en la relación final.
Por ejemplo, el acrónimo de API está definido en el fichero pero no lo 
se utilizado y por lo tanto no está en el PDF resultante

En cuanto a los términos del glosario, serán las definiciones de los
términos técnicos como puede ser un bytecode o un hash. Además también 
se debieran de incluir aquellos términos relacionados con el problema 
de especial relevancia.
Por ejemplo, piense en que esta realizando una  aplicación para la gestión 
del máster, un término que a cualquier lector tendrái que explicarsele es
el significado de \gls{credito} en el marco de la enseñanza universitaria.
Estos términos estarán definidos en \texttt{05\_terminos/02\_glo\-sa\-rio.tex} 
y se pueden invocar con la orden \texttt{gls} (por ejemplo, \gls{credito}) 
o bien \texttt{Gls} ( \Gls{credito}). Igual que los acrónimos, aquellos 
términos que no se usen en el texto no se incluirán en el listado final.

\section{Errores comunes}

Existen algunos errores que son sencillos de detectar pero que, en muchas ocasiones, 
al autor se le pasan inadvertidos por culpa de una lectura un tanto descuidada. 

\begin{itemize}
    \item Errores con las comillas. El autor queriendo escribir algo entrecomillado 
    como ``ejemplo de entrecomillado''. Acaba escribiendo algo similar a ''error 
    de entrecomillado'' o "error de entrecomillado". En ambos casos se debe que las 
    comillas se han aplicado de manera incorrecta siendo la forma correcta 
    \texttt{``ejemplo de entrecomillado''}.
    \item Errores debido la posición de barras bajas o comillas. Un autor despistado 
    puede acabar escribiendo $06_imagenes$ o algo peor en el párrafo en lugar de 
    06\_imagenes. El problema  viene de que el caracter \_ y el carácter " debieran 
    de escaparse con la barra ya que de otra manera tiene el significado de subindice 
    o superindice, respectivamente.
\end{itemize}