\chapter{Introducción}
\label{chap:introduccion}

\lettrine{N}{este} primeiro capítulo da memoria abordaráse a introducción á problemática a tratar. Ademáis, explicarase a motivación para levar a cabo o desenvolvemento do proxecto e por últimos comentaránse os obxetivos que se esperan acadar.

***********************************
A continuación se muestran un ejemplo de como referencias una sección 
en concreto del documento como puede ser la de objetivos del trabajo. 
En el presente documento, esta sección es la Sección \ref{sec:objetivos}
que podemos ver en la página \pageref{sec:objetivos}.

Por último dentro de este primer capítulo de ejemplo, se incluyen un 
par de citas~\cite{RAE,RAGA, van2005student} bibliográficas a modo 
de ejemplo. En este sentido, si los o las estudiantes piensan en 
escribir está memoria en en overleaf \cite{Overleaf}, se recomienda el 
consultar la integración con Mendeley \cite{Mendeley} ya que les puede 
facilitar el mantenimiento de las referencias.
***************************************

\section{Motivación}

Hoxe en día, existen diferentes comunidades en gremios, así como acontece co conxuntos dos músicos: un grupo nicho dentro da sociedade que buscan desenvolverse neste ámbito pero moitas veces resulta difícil debido á ampla cantidade de información que circula polas redes sociais e distintos foros. Aínda que cada vez esta información sexa personalizada, sigue habendo datos que pode pasar desapercibidos para certos colectivos, como pode ser o caso dos músicos, que poder perder información debido ao gran volume de datos na actualidade.

Tras consultar e compartir pensamentos con varios músicos e expertos no tema, analizouse que clase de información eles considerarían de relevancia e virían necesaria nunha proposta de aplicación enfocada e destinada ó colectivo de músicos, ou persoas interesadas en seguir eventos  musicais de todo tipo. Houbo concenso xeral na necesidade de dar a coñecer eventos locais ou pequenos, ou anuncios buscando xente para tocar en concertos que moitas veces pasaban desapercibidos en outras redes sociais máis grandes. Entre outras ideas, tamen mencionaban a necesidade de poder discutir en foros de discusións nos que expoñer e comentar o seu traballo, con outra xente de experiencia similar a eles, que calquera crítica constructiva lle sería de utilidade xa que existen moitos foros nos que persoas alleas ou pouco expertos no tema tenden a participar en debates sen fundamentos ou xustificacións non válidas.
En moitos casos, os encuestados compartían unha opinón xeral de ter un lugar no que almacenar todas as súas partituras e que foran accesibles desde calquer lugar. 

Polo tanto xurde así a idea desta aplicación: \textbf{\textit{Music Community}}. Partirase dun sistema xa establecido, que consta dun servicio REST e unha aplicaicón web, pero para o desenvolvemento deste traballo realizarase unha transformación ao entorno móbil. Xa que as tendencias cambian no tempo, e debemos adaptarnos as novas tecnoloxías. Isto representa un novo desafío a nivel académico e persoal, no que asentar os coñecementos obtidos no mestrado, realizando a transformación dunha aplicación legacy nun proxecto actualizado.

\section{Estado do arte}


\section{Objetivos}
\label{sec:objetivos}
Una de las secciones más habituales dentro de la memoria es la correspondiente a los objetivos del proyecto, los cuales se suelen separar en objetivos generales y objetivos concretos.  

\subsection{Objetivos principales}

\begin{enumerate}[label={\textbf{Objetivo \arabic*:}},leftmargin=2.5cm,labelindent=\parindent]
    \item Primer objetivo del proyecto
    \item Segundo objetivo del proyecto
    \item Los siguientes si los hubiere \ldots
\end{enumerate}

\subsection{Objetivos concretos}
\begin{enumerate}[label={\textbf{Objetivo Concreto \arabic*:}},leftmargin=4.5cm,labelindent=\parindent]
    \item Primero
    \item Segundo 
    \item Y así sucesivamente \ldots
\end{enumerate}


\section{Estructura de la memoria}
Otro de los elementos típicos dentro del primer capítulo de las memorias es designar las estructura de la misma con una brevísima  explicación de que se puede encontrar en cada capítulo.

\begin{enumerate}[label={Capítulo \arabic*},leftmargin=2.5cm]
\item Este mismo capítulo.
\item El segundo de los capítulo centra \ldots
\end{enumerate}

