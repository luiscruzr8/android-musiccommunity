\chapter{Introducción}
\label{chap:introduccion}

\lettrine{E}{n} el primero de los capítulos de la memoria, se espera 
que el alumno haga una introducción a la problemática a tratar. Esta 
debe de ser abordada desde un punto de vista general. Los elementos a
tratar durante está introducción, además de una idea general del campo,
serán aquellas líneas maestras del trabajo y los objetivos.

A continuación se muestran un ejemplo de como referencias una sección 
en concreto del documento como puede ser la de objetivos del trabajo. 
En el presente documento, esta sección es la Sección \ref{sec:objetivos}
que podemos ver en la página \pageref{sec:objetivos}.

Por último dentro de este primer capítulo de ejemplo, se incluyen un 
par de citas~\cite{RAE,RAGA, van2005student} bibliográficas a modo 
de ejemplo. En este sentido, si los o las estudiantes piensan en 
escribir está memoria en en overleaf \cite{Overleaf}, se recomienda el 
consultar la integración con Mendeley \cite{Mendeley} ya que les puede 
facilitar el mantenimiento de las referencias.

\Blindtext

\section{Objetivos}
\label{sec:objetivos}
Una de las secciones más habituales dentro de la memoria es la correspondiente a los objetivos del proyecto, los cuales se suelen separar en objetivos generales y objetivos concretos.  

\subsection{Objetivos principales}

\begin{enumerate}[label={\textbf{Objetivo \arabic*:}},leftmargin=2.5cm,labelindent=\parindent]
    \item Primer objetivo del proyecto
    \item Segundo objetivo del proyecto
    \item Los siguientes si los hubiere \ldots
\end{enumerate}

\subsection{Objetivos concretos}
\begin{enumerate}[label={\textbf{Objetivo Concreto \arabic*:}},leftmargin=4.5cm,labelindent=\parindent]
    \item Primero
    \item Segundo 
    \item Y así sucesivamente \ldots
\end{enumerate}


\section{Estructura de la memoria}
Otro de los elementos típicos dentro del primer capítulo de las memorias es designar las estructura de la misma con una brevísima  explicación de que se puede encontrar en cada capítulo.

\begin{enumerate}[label={Capítulo \arabic*},leftmargin=2.5cm]
\item Este mismo capítulo.
\item El segundo de los capítulo centra \ldots
\end{enumerate}

